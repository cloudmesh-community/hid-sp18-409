% status: 0
% chapter: TBD

\title{Location based crime data search and analysis with Spark UDF}


\author{Kadupitiya Kadupitige}
\orcid{1234-5678-9012}
\affiliation{%
	\institution{Indiana University}
	\streetaddress{Smith Research Center}
	\city{Bloomington} 
	\state{IN} 
	\postcode{47408}
}
\email{jasakadu@iu.edu}

\author{Gregor von Laszewski}
\affiliation{%
  \institution{Indiana University}
  \streetaddress{Smith Research Center}
  \city{Bloomington} 
  \state{IN} 
  \postcode{47408}
  \country{USA}}
\email{laszewski@gmail.com}


% The default list of authors is too long for headers}
\renewcommand{\shortauthors}{G. v. Laszewski}


\begin{abstract}
This paper provides a sample of a \LaTeX\ document which conforms,
somewhat loosely, to the formatting guidelines for
ACM SIG Proceedings.
\end{abstract}

\keywords{hid-sp18-409, Crime data analysis, Kmeans clustering, Google Maps, 
spark Mllib}


\maketitle


\section{Introduction}

A crime is generally defined as an act punishable by law and is one of
the dangerous factors for any
country\cite{hid-sp18-409-agarwal2013crime}. It is impossible to find
a country with a crime- free society due to complex causes such as
poverty, parental neglect, low self-esteem, alcohol, drug abuse and
etc~\cite{hid-sp18-409-bharathi2014survey,
hid-sp18-409-kiani2015analysis}.  Inspired by the big data revolution,
the historical way of crime solving and prevention has greatly
influenced and improved by the help of state-of-the-art data analytic
tools and machine learning research. Law enforcement officers have
started involving more data analytic expertises to speedup the crime
solving process highlighting that, it is an interdisciplinary research
area.

According to Chicago Police Department, crime prevention is important
and much more difficult to handle than crime solving for law
enforcement agencies~\cite{hid-sp18-409-www-cpd}. Law enforcement
officers could identify crime prone areas or suspicious activities
even before a crime is committed with the aid of pattern recognition
and big data analysis~\cite{hid-sp18-409-nath2006crime,
hid-sp18-409-gera2014city}.  According to the Los Angeles Police
Department (LAPD) and Chicago Police Department (CPD), whenever a
crime was committed somewhere, more crimes are likely to be occurred
in a nearby area, and the patterns of those criminal activities could
be identified through big data analysis~\cite{hid-sp18-409-www-cpd,
hid-sp18-409-www-lapd}. However, crime prevention could be so much
effective if there is an easy way to check the crime prone areas so
that people could be aware the of crime prone areas.

As inspired by the requirement to raise the awareness of crime prone
areas, we implement a location based crime search website where people
can find crimes happened near a particular geographical location (in a
vicinity area) for a given GPS coordinates. For this project, we have
selected a dataset which includes the crime information from 2001 to
current date in Chicago city~\cite{hid-sp18-409-www-data.gov}. We
further analyze this crime dataset using the crime type and timestamps to
identify the trends of crime types for a particular
geographical area. One applicational use of this project would be a
user trying to buy a new house and he may like to do a background
search on criminal activities around the geographical area for safety
concerns.

The organization of this project report is as follows. In section II,
a literature review on existing crime data analyses and crime
reporting web services are presented. Section III provides a
methodology followed in this project while section IV elaborates the
results and discussion. A summary about the project paper is provided
in section V.

\section{Literature Review}

\section{Methodology}

\section{Results and Discussion}

\section{summary} 

\begin{acks}
	
The authors would like to thank Dr.~Gregor~von~Laszewski for his
support and suggestions to write this extended abstract.
	
\end{acks}

\bibliographystyle{ACM-Reference-Format}
\bibliography{report} 
